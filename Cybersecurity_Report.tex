                        Cybersecurity Reports.

                            2016 Election.

CNN provided us with 2016 Presidential Campaign Hacking Fast Facts as material to help us report about cyber current events.

In the report we highlight the type of attack, how the vulnerability was discovered, how the vulnerability was exploited by attackers and some security measures taken to prevent this attack.
In the 2016 US election, (CNN) reported about the Phishing attack that succeeded in stealing information and emails.

What is Phishing?

Phishing is an attempt by an individual or group to solicit personal information from unsuspecting users by employing social engineering techniques. Phishing emails are crafted to appear as if they have been sent from a legitimate organization or known individuals. The users are asked to click on a link that will take them to a fraudulent website that appears legitimate. The users may be asked to provide personal information, such as account usernames and passwords, that can further expose them to future compromises.

This is exactly what happened in the 2016 election when Hillary Clinton’s chairman John Podesta received a phishing email that was marked as an alert from Google that another user was trying to access his account. John reported to the staffer from the campaign’s help desk who replied with a typo that “This is a legitimate email” instead of typing “This is an illegitimate email.” Podesta followed the instructions and typed a new password; the hackers can now access his emails. This was how the vulnerability was discovered.

From there a fake email account was created by hackers that was used to send spear-phishing emails to many Clinton staffers. The hackers embedded a link in the emails pretending to direct the recipient to a document, the link directed the recipients’ computers to a website operated by the hackers. They use stolen credentials to access the Democratic Congressional Campaign Committee computer network, stealing data with malware. This is how the vulnerability was exploited.

The hackers were from Russia whose role was to boost Trump and hurt Clinton, and an executive order with sanctions against Russia had been issued, and some order measures occurred upon investigations. Several recommendations were directed to President Trump. Also, a joint statement calling on Congress to work on securing future elections and stopping cyberattacks.

                        Broadvoice Data Breach.

We report about the cyber current event of Broadvoice data breach provided by TechRadar.

What is a Data Breach?

To define data breach: a data breach exposes confidential, sensitive, or protected information to an unauthorized person. The files in a data breach are viewed and/or shared without permission.

This is exactly what happened here, a cluster of databases containing over 350 million customer records such as names, phone numbers, call transcripts include voicemails, and other personal information was left open online, without a password required to access it, as discovered by security expert Bob Diachenko, working on behalf of Comparitech.

The hackers can exploit this vulnerability as follows: the leaked database represents a wealth of information that could help facilitate targeted phishing attacks, at this point if cyber criminals get hold of this personal information, they can use it to access things like your bank and online accounts.

No one so far can say with any certainty if the leaked data was accessed, therefore some steps have been taken:

Broadvoice acted fast to patch the security flaw.

The relevant legal authorities have been notified.

They launched an investigation and ensured the data had already been secured.

They alerted federal law enforcement and offered their full cooperation.

They worked with the security researcher to ensure that the data he accessed was destroyed.

A third-party forensics firm was engaged to analyze this data, afterward provide more information and update to their customers and partners.

No further speculation about the issue currently.
